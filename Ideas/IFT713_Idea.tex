\documentclass[a4paper ,10pt]{article}

\usepackage[french]{babel}
\usepackage[T1]{fontenc}
\usepackage[left=25mm,right=25mm,top=20mm,bottom=20mm,paper=a4paper]{geometry}
\usepackage[pdftex,
            pdfauthor={Dorian Gilbert, Mazen Ben Hmida, Nour El Houda Taouali, William Lapointe, Al Sadick Ismail Altoum},
            pdftitle={Idée de projet pour IFT605/713},
            pdfkeywords={Idée de projet pour IFT605/713}]{hyperref}
\usepackage{soul}

\title{Idée de projet pour IFT605/713}
\date{\today}
\author{Dorian Gilbert - Mazen Ben Hmida - Nour El Houda Taouali - William Lapointe - Al Sadick Ismail Altoum}

\begin{document}
\pagenumbering{gobble}

\section*{Idée de projet pour IFT605/713}

\subsection*{Idée 1 : Maison intelligente}

\subsubsection*{Objectif du projet}
Créer un système IoT intégré qui :

\begin{itemize}
  \item Surveille la sécurité de la maison en détectant les intrusions à travers des capteurs de mouvement et de porte.
  \item Gère automatiquement l’éclairage en fonction de la présence et de l’heure de la journée.
  \item Surveille les niveaux d'eau pour prévenir les fuites et inondations.
  \item Envoie des notifications en temps réel pour tous les événements importants (intrusion, éclairage automatique, fuite d'eau) via une application mobile ou une plateforme web.
\end{itemize}

\subsubsection*{Matériel nécessaire}

\begin{enumerate}
  \item Raspberry Pi 3B : Utilisé comme unité centrale de traitement pour gérer les différents capteurs et contrôler le système.
  \item Capteurs de Mouvement (Zwave Plus Motion Sensor) : Pour détecter la présence humaine et contrôler l'éclairage ou signaler une intrusion.
  \item Capteurs de Porte/Fenêtre (Zwave Plus Door/Window Sensor) : Pour détecter si une porte ou une fenêtre est ouverte, et déclencher une alerte en cas d'intrusion.
  \item Capteur d'Inclinaison pour Porte de Garage : Pour vérifier si la porte de garage est ouverte ou fermée.
  \item Ampoules LED intelligentes (Zwave Plus LED Bulb) : Pour gérer l'éclairage automatique en fonction de la détection de mouvement et des conditions définies.
  \item Détecteurs de niveau d'eau : Pour surveiller les fuites ou les inondations potentielles.
  \item Bouton Fibaro : Pour activer/désactiver manuellement le système ou activer un mode "sécurité".
\end{enumerate}

\subsubsection*{Fonctionnalités du Projet}

\begin{enumerate}
  \item Sécurité de la Maison
  \item Gestion Automatique de l'Éclairage
  \item Surveillance des Niveaux d'Eau
  \item Système d'Alerte et Interface Utilisateur
\end{enumerate}

\newpage

\subsection*{Idée 2: Système d’Irrigation Intelligent pour l’Agriculture}

\subsubsection*{Objectif du projet}
L'objectif de ce projet est de développer un système d'irrigation autonome pour surveiller et contrôler l'humidité du sol d'un champ/jardin, assurant ainsi une irrigation optimale et prévenant le gaspillage d'eau.

\subsubsection*{Matériel nécessaire}

\begin{enumerate}
  \item Détecteurs de niveau d'eau et capteurs d'humidité : Mesurent l'humidité du sol en temps réel.
  \item Zwave Plus Motion Sensor : Détecte l'activité autour de la zone, utile pour surveiller les passages d’animaux ou d’ouvriers pour une gestion de sécurité en parallèle.
  \item LED Diffusées Vertes et Rouges : Indiquent visuellement l'état d'irrigation (vert pour un sol suffisamment humide, rouge pour un besoin d'irrigation).
  \item Raspberry Pi 3B : Serveur central qui collecte les données des capteurs, exécute la logique de décision et envoie les commandes d'irrigation.
  \item Zwave USB Z-Stick : Connecté au Raspberry Pi pour gérer la communication avec les capteurs Zwave.
  \item Pompe à eau et réservoir (à simuler) : Pour fournir l'eau au champ/jardin.
  \item Bouton Fibaro : Permet d’activer ou de désactiver manuellement le système.
\end{enumerate}

\subsubsection*{Fonctionnalités du Projet}

\begin{enumerate}
  \item Surveillance en temps réel de l'humidité du sol : Les détecteurs mesurent l'humidité et envoient les données au Raspberry Pi à intervalles réguliers.
  \item Déclenchement automatique de l'irrigation : Si le niveau d'humidité tombe en dessous d'un certain seuil, la pompe à eau est activée pour irriguer le sol jusqu'à ce que l'humidité atteigne un niveau optimal.
  \item Alerte visuelle :
        \begin{itemize}
          \item La LED verte s'allume lorsque l'humidité du sol est dans la plage optimale.
          \item La LED rouge s'allume si l'humidité est en dessous du seuil minimum, indiquant un besoin d'irrigation.
        \end{itemize}
  \item Historique et analyse des données : Les données d'humidité et d'irrigation sont stockées et peuvent être analysées pour optimiser les cycles d'irrigation.
  \item Notifications et rapports : Envoi de notifications par SMS ou email ou application mobile/web si l'humidité descend sous le seuil critique ou si un problème est détecté (par exemple, panne de la pompe).
  \item Mode manuel : L'utilisateur peut utiliser le bouton Fibaro pour activer ou désactiver le système manuellement en cas de besoin particulier.
\end{enumerate}

\end{document}